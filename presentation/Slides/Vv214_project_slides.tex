\documentclass[12pt, t]{beamer}
\usepackage{graphicx}
\usepackage{amsmath}
\usepackage{setspace}
\usepackage{float} 
\usepackage{multido}
\usepackage{multirow}
\usepackage{array}
\usepackage{enumerate}
\usepackage{booktabs}
\usepackage{indentfirst} 
\usepackage[style=mla]{biblatex}
\usepackage{subcaption}
\usepackage{hyperref}
\usepackage{textpos}
\usepackage{mathtools, nccmath}

\makeatletter
\let\@@magyar@captionfix\relax
\makeatother

\definecolor{Turquoise3}{RGB}{0, 134, 139}
\renewcommand{\emph}[1]{{\color{Turquoise3}\textsl{#1}}}
\newcommand{\C}{\mathbb{C}} \newcommand{\F}{\mathbb{F}} \newcommand{\R}{\mathbb{R}} \newcommand{\Q}{\mathbb{Q}}
\newcommand{\N}{\mathbb{N}}
\newcommand{\myseries}[2]{$#1_1,#1_2,\dots,#1_#2$}
\newcommand{\nullspace}{~\\[15pt]}
\newcommand{\remark}{\textbf{Remark: }}
\newcommand{\scp}[2]{\langle\,#1\,,\,#2\,\rangle} \newcommand{\scpp}{\langle\,\cdot\,,\,\cdot\,\rangle}


\usetheme{Madrid}
\setbeamertemplate{navigation symbols}{}

\addtobeamertemplate{frametitle}{}{
\begin{textblock*}{100mm}(0.85\textwidth,-1cm)
\includegraphics[height=1cm]{Figures/logo/logo.png}
\end{textblock*}}

\definecolor{themecolor}{RGB}{25,25,112} 

\usecolortheme[named=themecolor]{structure}

\setbeamertemplate{items}[default]

\hypersetup{
    colorlinks=true,
    linkcolor=themecolor,
    filecolor=themecolor,      
    urlcolor=themecolor,
    citecolor=themecolor,
}

\title{Vv214 Final Project}
\subtitle{\textbf{Donut.c}}
\institute[UM-SJTU JI]{University of Michigan-Shanghai Jiao Tong University Joint Institute}
\author{Pingbang Hu,  Xiaoyu Chen, Jinyi Wu}

\begin{document}

\begin{frame}
    \titlepage
    \begin{center}
        \includegraphics[height=2cm]{Figures/logo/logo2.png}
    \end{center}
\end{frame}

%--------------------------------------------------------------------------------------------------
\section{Overview}
    \begin{frame}
        \frametitle{Overview}
        \begin{enumerate}
            \item Motivation
            \item Introduction
            \item Draw a donut
            \item Rotate a donut
            \item Bright and dark
            \item Projection into terminal
            \item Discrete Dynamic System
            \item Extra
            \item Source Code
            \item Reference
        \end{enumerate}
    \end{frame}

%--------------------------------------------------------------------------------------------------

\section{Motivation}
\begin{frame}
    \frametitle{Motivation}

\end{frame}

%--------------------------------------------------------------------------------------------------

\section{Introduction}
\begin{frame}
    \frametitle{Introduction}

\end{frame}

%--------------------------------------------------------------------------------------------------

\section{Draw a donut}
\begin{frame}
    \frametitle{Draw a donut}

\end{frame}

%--------------------------------------------------------------------------------------------------

\section{Rotate a donut}
\begin{frame}
    \frametitle{Rotate a donut}

\end{frame}

%--------------------------------------------------------------------------------------------------

\section{Bright and Dark}
\begin{frame}
    \frametitle{Bright and Dark}

\end{frame}

%--------------------------------------------------------------------------------------------------

\section{Projection into terminal}
\begin{frame}
    \frametitle{Projection into terminal}

\end{frame}

%--------------------------------------------------------------------------------------------------

\section{Discrete Dynamic System}
\begin{frame}
    \frametitle{Discrete Dynamic System}

\end{frame}

%--------------------------------------------------------------------------------------------------

\section{Reference}
\begin{frame}
    \frametitle{Reference}

    \small{
        \begin{itemize}
            \item \url{https://www.a1k0n.net/2011/07/20/donut-math.html}
            \item \url{https://en.wikipedia.org/wiki/3D_computer_graphics}
            \item \url{https://www.javatpoint.com/computer-graphics-z-buffer-algorithm}
        \end{itemize}
    }
    
\end{frame}


\end{document}